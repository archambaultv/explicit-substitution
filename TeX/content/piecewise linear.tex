% !TEX root = ../main.tex

\section{Piecewise linear function}\label{sec:pwl}

We review in this section the basic mathmatical facts about piecewise linear
functions that are needed for this work. Normally such functions are defined on
the real numbers, but it is sufficient here to defined them on the natural
numbers.

\begin{definition}
  \label{def:PW}
  A function $f : \mathbb{N} \rightarrow \mathbb{N}$ is said to be piecewise
  linear if there is a finite set $Q$ of intervals such that $Q$ is a partition
  of $\mathbb{N}$ and $f$ is linear on every interval in $Q$.
\end{definition}

\begin{definition}
  \label{def:PW_support}
  A linear support is a finite set of pairs $(q,g)$ such that $q$ is an interval
  of $\mathbb{N}$ and $g$ is a linear function. Also, the set of interval $q$ of
  each pair forms a partition of $\mathbb{N}$.
\end{definition}
Since the intervals of a linear support are a partition of $\mathbb{N}$, we
abuse the function notation and write $Q(x)$ for the value of the function $g$
paired with the interval $q$ such that $x \in q$. Also, we say that $Q$ is the
linear support of $f$ if $f(x) = Q(x)$ forall $x$.

It is obvious from those definition that any piecewise linear function has a
linear support and that a linear support defines a unique piecewise linear function.
Note that the linear support of a piecewise linear function is not unique.
\begin{lemma}
  \label{thm:support_unique_pl}
  A linear support defines a unique piecewise linear function.
\end{lemma}
\begin{lemma}
  If $f$ is a piecewise linear function, then $f$ has a linear support.
\end{lemma}

\subsection{Composition of piecewise linear functions}



\begin{lemma}
  The composition $f = f_2 \circ f_1 $ of two piecewise linear functions $f_1$ and $f_2$ is also a
  piecewise linear function. Its support $Q$ can be computed from the support
  $Q_1$ and $Q_2$ of $f_1$ and $f_2$.
\end{lemma}
\begin{proof}
  By lemma \ref{thm:support_unique_pl} we only need to show how to compute its
  support. For each interval $[x,y]$ in $Q_1$ and its component $g$, since $g$
  is linear we have an image interval $[x',y']$ where $x' = g(x)$ and $y' =
  g(y)$ if $g(x) \le g(y)$ or $x' = g(y)$ and $y' = g(x)$ in the case $g(x) >
  g(y)$.

  Let $\Gamma$ be the set of all overlaps between an interval in $Q_2$ and the
  interval $[x',y']$. 
  \begin{equation*}
    \Gamma = \{q \cap [x',y'] \ | \ q \in Q_2 \land q \cap [x',y'] \neq \emptyset \}
  \end{equation*}
  It is clear that $\Gamma$ is a set of intervals that is a partition of
  $[x',y']$. Again, since $g$ is linear we can inverse the intervals in $\Gamma$
  to expressed them intervals in the domains of $g$. 
  \begin{equation*}
    \Gamma^{-1} = \{[g^{-1}(x), g^{-1}(y)] \ | \ [x,y] \in \Gamma \}
  \end{equation*}
  $\Gamma^{-1}$ is a set of intervals that is a partition of $[x,y]$. The value
  of $f$ on any point $x \in [x,y]$ is $f(x) = (g_2 \circ g)(x)$ where $g_2$ it
  the component of $f_2$ associated with the interval 

  
  We can repeat this process for each element of $Q_1$. The union of all the
  $Gamma$ at each iteration is the set $Q$. It is made only of interval since
  each set $\Gamma$ at any iteration is limited to the interval

\end{proof}

% The backbone of a substitution is a piecewise linear function with all slopes
% equal to one. We use the abbreviation $\text{pwl}$ to denote such functions.
% Such a piecewise linear function can be characterized by its points of
% discontinuity, where they are and the offset relative to the next segment. As
% such, this kind of piecewise linear function can be represented by a list of
% tuples $[(i_1,k_1),(i_2,k_2)\ldots(i_n,k_n)]$. The meaning of the tuple $(i, k)$
% is that the "jump" by $k$ occurs \emph{at} the value $i$. When the context is
% clear, we simple write $i_n$ and $k_n$ the first and second parameters of the
% ith tuple in the list.

% We call such a list of tuples the support of the function. To make the
% representation unique, we ask that the list is well-formed with ordered tuples
% and with all non zero offset.  Note that, We can always take an arbitrary list
% of tuples and compute an well-formed one equivalent. Also, the identity function
% is encoded as the well-formed support $[]$.

% \begin{equation*}
%   \text{pwl} \equiv [(i_1,k_1),(i_2,k_2)\ldots(i_n,k_n)]
% \end{equation*}
% \begin{equation*}
%   \text{wf} \ [(i_1,k_1),(i_2,k_2)\ldots(i_n,k_n)] \leftrightarrow 
%     (\forall n \, m, n < m \rightarrow i_n < i_m)  \land \forall n, k_n \neq 0 
% \end{equation*}

% From its reprensentation, we can easily compute the value of any piecewise linear function $f$.
% \begin{gather*}
%   \text{f} \ x = \text{lookup} \ x \ [(i_1,k_1),(i_2,k_2)\ldots(i_n,k_n)] \\
%   \text{lookup} \ x \ [] = x \\
%   \text{lookup} \ x \ [(i_1,k_1),(i_2,k_2)\ldots(i_n,k_n)] = 
%   \begin{cases}
%     x \quad \text{if $x < i_1$} \\
%     k_1 + \text{lookup} \ x \ [(i_2,k_2)\ldots(i_n,k_n)] \quad \text{otherwise} 
%   \end{cases}
% \end{gather*}

% Two piecewise linear functions are equal if their well-formed support are equal.
% This is an important property of well-formed supports, because it means we can
% compute the extensional equality of function by computing the intensional
% equality of their supports.

% \subsection{composition}\label{sec:pwl_composition}
% \begin{align*}
%   \text{compose}& \ [] \ f_2  ={} f_2 \\
%   \text{compose}& \ f_1 \  [] ={} f_1 \\
%   \text{compose}& \ ((i_1,k_1) :: x_1) \ ((i_2,k_2) :: x_2) ={} \\ 
%     &\begin{cases}
%        \quad \text{if $i_1 < i_2$} \\
%       \quad \text{if $i_1 = i_2$} \\
%       (i_2, k_2) :: \text{compose} \ ((i_1,k_1) :: x_1) \ x_2 \quad \text{if $i_1 > i_2$} \\
%     \end{cases}
% \end{align*}