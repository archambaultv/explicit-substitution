% !TEX root = ../main.tex

\section{Substitution}\label{sec:substitution}

A piecewise linear function, of type $\mathbb{N} \to \mathbb{N}$ can be extended
to work on terms of the lambda calculus by viewing it as a function of variables
to variables. So piecewise linear function are basically renaming functions.
They map variables to other variables. A substitution is basically an extended
piecewise linear function except that for some value, it returns an arbitrary
terms and not just a variable.

But a substitution is not any kind of function $\mathbb{N} \to \lambda$. By
analogy with the section \ref{sec:pwl}, we will formaly define a substitution by
requiring that there must exists a finite set of components that describe it.
\begin{definition}
  \label{def:subst}
  A function $s : \mathbb{N} \to \lambda$, is said to be a substitution if there
  exists a finite map $m$ from $\mathbb{N}$ to $\lambda$ and a piecewise linear
  function $l$ such that $s = \# \circ l$ on $\mathbb{N} \setminus
  \text{dom}(m)$. The map $m$ and the function $l$ forms the pair of components
  of $s$.
\end{definition}
Note that from this definition, the pair of components $(m, l)$ for a
substitution $s$ is not unique. 

\subsection{Propagation of substitutions}\label{sec:subst_propagation} 

A substitution is a function from variables to terms. In order to apply a
substitution to a term, we need to propagate it through the term in order to
reach the variables. In doing so, we need to take into account the binding
introduced by lambda abstraction. In order to do so, we need to introduce a
special operator $\liftop$, pronounced lift. 
\begin{definition}
  The operator $\liftop$ takes a natural number $n$, a function
  $f : \mathbb{N} \to \lambda$ and returns a function of type $\mathbb{N} \to
  \lambda$ such that
  \begin{equation}
    (\liftop(n, f))(x) = \begin{cases}
      \#x \quad \text{if $x < n$} \\
      \shift[n] (f(x - n)) \quad \text{if $x \ge n$}
    \end{cases}
  \end{equation}
  We write $\lift[n]$ for the partial application of $\mathcal{L}$ with the number $n$.
\end{definition}

\begin{lemma}
  For all $f$, we have $\lift[0](f) = f$.
\end{lemma}
\begin{lemma}
  For all $f$, $n$ and $m$, we have $\lift[n](\lift[m](f)) = \lift[{n+m}](s)$.
\end{lemma}
\begin{lemma}
  If s is a substitution, then $\mathcal{L}_{n}(s)$ is also a substitution.
\end{lemma}

We can now extend the definition of a substitution from a function of variables
to terms to a function of terms to terms.
\begin{equation}
  t[\sigma] = \begin{cases}
    \sigma(i) \quad \text{if $t = \#i$} \\
    \lambda (t_1[\lift ({\sigma})]) \quad \text{if $t = \lambda t_1$} \\
    (t_1[\sigma]) \ (t_2[\sigma]) \quad \text{if $t = t_1 \ t_2$}
  \end{cases}
\end{equation}

\subsection{Composition of substitutions}\label{sec:subst_composition}

Substitutions are functions from natural numbers to terms. As such they don't
compose since their domain do not match their codomain. But we can defined the
composition $s_2 \circ s_1$ as the substitution $s$ such that $(t[s_1])[s_2] =
t[s]$ holds for any $t$. The following lemma shows that $s$ always exists for
any $s_1$ and $s_2$.

\begin{lemma}
  The composition $s = s_2 \circ s_1 $ of two substitution $s_1$ and $s_2$
  always exists and is also a substitution. A pair of components $(m, l)$ can be
  obtained from the pairs of components $(m_1, l_1)$ and $(m_2, l_2)$ of $s_1$
  and $s_2$.
\end{lemma}
\begin{proof}
  
\end{proof}